\documentclass{article}
\usepackage[utf8]{inputenc}

\title{Abstract}
\author{nocturnalbeast}
\date{February 2019}

\usepackage{biblatex}
\addbibresource{cite.bib}

\begin{document}

\maketitle

KEYWORDS: smartphone addiction, mental health, brain waves

\paragraph{} The smartphone is a wondrous device, and it's usage has been steadily on the rise since the turn of the last decade. However, the purported benefits of this device and the increase in productivity it offers is not without its caveats. Several studies have proven that increase in smartphone usage have been linked to deterioration of mental health\cite{bian2015linking,twenge2017have,ward2017brain}. It is also found that the mental health of smartphone users have been declining on an average. Thus, it comes as no shock when we understand that the pervasive adoption of these devices without the user being aware of such risks drags an increasing amount of people into mental health issues like depression and lethargy. This paper attempts to tackle the aforementioned problem by proposing a solution which aims to decrease such excessive usage of smartphones amongst users. It is a smartphone application which connects to a wearable system which tracks the user's mental acuity and status, and helps the user limit smartphone use by suggesting a daily limit on the time he/she is able to use the device. This application proves to be more effective than its counterparts utilising similar strategies to reduce smartphone usage, since it delivers dynamic and personalized usage limits based on the multiple inputs recorded by the wearable system that it connects to. The solution proves to be effective in curbing excessive use, which resulted in better mental acuity, concentration, satisfaction and happiness in general amongst the users which tested the application. This was confirmed with 90.71\% of the people (381 out of 420 people) responding that they felt an improvement in their state of mind on a day-to-day basis after utilizing the app, when they were asked questions regarding the same on an exit interview conducted on the testing of the solution.

\newpage
\printbibliography

\end{document}

\documentclass{article}
\usepackage[utf8]{inputenc}

\title{Proposed Solution}
\author{nocturnalbeast}
\date{January 2019}

\begin{document}

\maketitle

\paragraph{} Considering the problem at hand, which was the excessive usage of smartphones amongst its users and the mental health risks it entails, the author decided to implement a multi-pronged solution to approach and provide a solution in two ways; to understand the patterns of smartphone usage and analyze the negative mental and behavioral patterns that ensue, and to provide a discretionary control mechanism to the user to limit the usage of a smartphone what can be considered acceptable usage durations without succumbing to addictive tendencies. The solution is a biometric system that consists of a wearable headband and a wearable wristband. The wearable headband records five kinds of brain waves, which are then used to understand the level of mental acuity that the user currently possesses, and the emotional state that the user is in. The wearable wristband is used to measure the frequency of touchscreen interactions on the smartphone which are performed by the user. Both the headband and the wristband are to be worn by a subject, and they are intended to be connected to the subject's smartphone, in which an mobile application that interfaces with the two wearables in question. The mobile application performs three activities; it interfaces with the wearable devices and pulls the sensor data from the devices and tracks usage of the smartphone over time, sends this data over to a central server for advanced analysis, and also provides an option to lock out the user from using the smartphone for a definite timeframe (which will be referred to as a lock-out timeframe in the coming sections).

\paragraph{} This solution has the potential to be an effective method in curbing the excessive usage of smartphones in users of these devices, since it is not just an application that renders a phone useless on a usage cycle that is deemed excessive, but this system tracks the mental acuity and emotional status of the user and relatively adjusts the application's limiting parameters to dynamically provide lock-out timeframes, which is a more persistently accurate detection of the user's negative behavioral cycles due to the excessive usage of such devices, which is inherently better than only providing a lock-out timeframe with a static time period. It also has the capability to monitor the user's behavioral and emotional trends over time and can suggest shorter or longer lock-out timeframes according to the need. These advanced detection and analysis methods will have an edge over statically-set lock-out timeframes and automatically takes into account the variations in emotional states over a long period of time, which is not possible to work with using a statically-set lock-out timeframe.

\paragraph{} The experiment, however is quite different from the solution. Rather than trying to limit users from excessive usage, the experiment tries to find the negative mental health effects that are caused by prolonged usage of smartphones. This is achieved by this experiment being run on a multi-ethnographic userbase, who will use the wearable devices alongside the application for an extended period of time. However, one group will have the lock-out mechanism enabled, while the second group will have the lock-out mechanism disabled. The disparities in behavioral patterns is then to be analysed and thus a result that either supports or rejects the premise is obtained.

\paragraph{} This solution so proposed plays an important part in the experiments that are to be conducted to prove that smartphones can cause detrimental effects on mental health over the course of excessive usage cycles. The data collection built into the application serves as the starting point into the experiment, and provides the necessary data which the experiment is supposed to generate. The proof of the premise then, can be proved or disproved using the given results obtained out of analyzing the generated data.

\paragraph{} The solution and the accompanying experiment is one of many in the field of HCI studies wherein the link between negative effects to mental health and the utilization patterns of smartphones have been extensively designed and carried out. However, the author believes that this is a novel idea amongst its peers, as there is no related literature that attempts to do what the author is trying to do here.

\end{document}

\documentclass{article}
\usepackage[utf8]{inputenc}

\usepackage{biblatex}
\addbibresource{cite.bib}

\title{Experiment Conducted}
\author{nocturnalbeast}
\date{February 2019}

\begin{document}

\maketitle

\section{Introduction}
\paragraph{}To understand and further analyze the hypotheses that excessive and/or persistent use of smartphones serve as a catalyst in development of mental health disorders, the author proposes an experiment that integrates into the usage of the proposed solution; as will be seen further in the details. The aim of the experiment is to prove that levels of usage of smartphones that reach signs of addiction to the former results in sub-par mental health, when compared to usage of smartphones that are restricted to safe levels; from which the given hypothesis can be inferred. This is accomplished by conducting a study on a multi-ethnographic subject groups, who are subjected to varying levels of smartphone usage and analyzing the recorded data. The experiment is conducted under the assumption that the usage levels being referred to on a temporal basis recorded each day, and as such multiple levels of time periods of smartphone usage are used as variant groups, which are further discussed below.

\section{Research Data}
\paragraph{}The experiment focuses on the variance in biological parameters pertaining to mental health, and as such is done with the help of data that is generated from continued usage of the smartphone application designed alongside the wearables that connect to it, which are discussed in detail later. The data collected include the following:

\begin{itemize}
    \item Brain waves, which include alpha waves, beta waves, gamma waves, delta waves and theta waves
    \item Eye movements
    \item Touchscreen interaction patterns and frequencies
\end{itemize}

\paragraph{}The above data is collected from every participant of the study, from usage of the testing system for the entire duration of the participant's usage of his/her smartphone. The analysis of data collected thus is done with the research that this experiment builds upon, as guidelines\cite{buzsaki2006rhythms}, given here:

\begin{itemize}
    \item Alpha waves correlate to the brain's state to learn and stay attentive.
    \item Alpha waves also correlate to tasks that require high degree of coordination.
    \item Beta waves correlate to the brain's state of decision making.
    \item It also acts as another indicator of the focused state of mind.
    \item Gamma waves indicate the heightened state of mind i.e. a trance-like state or a mindful state of deep meditation.
    \item Delta waves indicate the state of 'drifting away' or 'dreaming' and also indicates the urge to sleep.
    \item Theta waves are indicators of multiple things; sleep, learning and heightened states can all be indicated with the same.
    \item The speed of eye movements provides a temporal scale on the mental energy levels of the subject.
    \item The frequency of eye movements provides the relative scale of the attention span of the subject.
    \item Touchscreen interaction patterns are utilized to confirm the findings that are indicated with the given levels of brain waves at a certain instant of time.
\end{itemize}

\section{Experimental Setup}
\paragraph{}The framework for the experimental setup was provided by the proposed solution itself, thus allowing the solution system to act as the central setup for data generation, with which the aforementioned data was collected from the participants. The setup includes an Android application that runs on the participant's smartphone, and tracks the usage of the participant over time, and also maintains a history of the usage patterns for a defined number of days (which in this case was made to be the duration of the experiment i.e. 60 days). Additionally, the system also contains two wearables; a headband and a wristband which record biometric data such as brain waves and touchscreen interaction patterns. Eye movement data was collected using the smartphone itself, whose front-facing cameras, when equipped with the eye-tracking libraries integrated into the application, provided the necessary data. Overall, this system provided two kinds of data; a timeline of data that covers usage patterns of the user and the relevant aforementioned data to verify the hypothesis.

\section{Research Design}
\paragraph{}The experiment was conducted on a group of 630 participants, which are divided into three groups of three; a control group and two variant groups. The control group was not allowed access to the usage limiting function of the application, while the second variant group was allowed access to the usage limiting function of the application, and were limited to a maximum usage period of five hours per day, and the third variant group was allowed access to the usage limiting function of the application, and were limited to a maximum usage period of two hours per day. Thus, each group consisted of 210 participants. All the participants were in the age range of 18-50, out of which 54\% were men and 46\% were women. In matters of ethnicity, the participants contained 27\% North American participants, 49\% Asian participants, 12\% African participants and 10\% Australian participants. Each participant was allowed to use their smartphones as long as the application let them i.e. those without a set lockout timeframe were allowed unrestricted access and those with a set lockout timeframe were only allowed access until the lockout was triggered. The usage patterns and the data mentioned above were collected over the time period of 60 days from all the participants. 

\section{Data Generated}
\paragraph{}The two kinds of data were collected from each participant to generate graphs based on time, detailing the variations in brain waves and ones detailing the variations in usage patterns, which are backed by data from the touchscreen interaction data. In summation, a total of four graphs were generated for each participant:

\begin{itemize}
    \item the variation in the five kinds of brain waves,
    \item the variation in frequency and speed of eye movements,
    \item the usage patterns and the touchscreen interaction data, and
    \item total time of usage across each day,
\end{itemize}

\paragraph{}all of which were plotted over the duration of the whole experiment. The results were pooled together into groups according to which group each participant belonged to, and then they were collated together, accounting for variance (with general standard deviation functions) and then the collated results of each group were compared against one another to obtain the final results, which would be the instrumental data that would then prove/disprove the hypothesis.

\newpage
\printbibliography

\end{document}

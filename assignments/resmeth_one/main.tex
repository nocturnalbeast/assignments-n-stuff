\documentclass{article}
\usepackage[utf8]{inputenc}

\usepackage{biblatex}
\addbibresource{cite.bib}

\title{Vegetarianism is overrated}
\author{nocturnalbeast}
\date{January 2019}

\begin{document}

\maketitle

\section{Introduction}
\paragraph{} Vegetarianism usually refers to the abstinence of an individual from certain kinds of animal flesh and by-products such as poultry and dairy, which is further specifically detailed in its variants such as pollo-vegetarianism, lacto-vegetarianism, ovo-vegetarianism, pescetarianism and flexitarianism. Vegetarianism has been seen to be a life choice that is now seeing a very steady increase in its adoption due to reasons like purported health benefits, resistance to lifestyle diseases and the issues on ethics on the slaughter and ill-treatment of animals. Even though the promise of better health is alluring to many people, there are lots of issues regarding vegetarianism and its variants that people are not aware of, and thus it is given more credit than it deserves.

\section{Proposition}
\paragraph{} The health problems that arise due to ill-advised methods of vegetarianism is to be known first. It is to be understood that vegetarianism is a choice which is adopted by many individuals due to the health benefits that a vegetarian diet proclaims over a meat-inclusive one. While this is perfectly valid for a well-balanced and a well-planned vegetarian diet, the reality points toward another direction. Studies conducted on actual meals that vegetarians consume point out that only a small portion of the demographic have a well-planned and balanced diet. This leads to deficiencies in nutrients in vegetarians, which is the next health problem, caused by the one mentioned before. Meat contributes well to a good portion of nutrients, and while such nutrients can be sourced from certain plant-based food, they are harder to do so, and thus end up usually being missed out on. Nutrients such as Omega-3 fatty acids, Vitamin B12, creatine, calcium and zinc end up being missed out on due to the same. The next glaring health problem is surprisingly a mental health issue. People who quit meat cold turkey and jump onto a vegan diet are very susceptible to develop mental health issues like depression, mood swings and a feeling of constant lethargy. There is also the obvious risk of sourcing fruits and vegetables which are grown in environments where pesticides are used in large quantities, which is a practice that many well-known agricultural houses engage in. An uncommon, but still relevant health disorder which is not a direct product of vegetarianism is that the eating disorders in certain individuals camoflauge themselves as a need to strictly adhere to a vegetarian diet.

\paragraph{} Secondly, the questions raised against consumption of meat by people who follow some form of vegetarianism is a relevant issue that is to be addressed. The questions of ethics raised such are almost always blown out of proportion. Moreover, these questions are raised by people who possess a fervent adherence to vegetarianism and blindly believe that consumption of meat is inherently "evil". One very relevant example is a pro-vegan food documentary titled "What The Health" released in 2017, which was all over the media for the wrong reasons. It uses absurd statements and wildly inappropriate statistics to try and convince the watcher that veganism is the best way to go.\cite{wthdoc} This is just one example. Ill-advised families tend to force their children to adopt some form of vegetarianism, usually the one that the parents follow. This is counter-intuitive, since the purported health benefits of the vegan diet is not recieved well with children, since they need nutrients that are primarily found in meats such as protein, DHA, creatine and so on. Even though some people who adopt vegetarianism are strictly against the killing of animals for food, they are decidedly ignorant of the food chain in nature. They seem to ignore the fact that there is a certain progression to how the predator of a prey becomes food to another. Some people even go to some extremes that they bad mouth people that consume meat and consider themselves as superior.

\paragraph{} Thirdly, we see that the environmental concerns raised by vegetarianism is also of significance. The constant use of pesticide in large-scale agricultural houses poisons the food and the planet as well. Another issue is the deforestation and habitat loss that is brought about by industrial agriculture\cite{vegenv2}, which requires large areas for crop cultivation and thus agricultural houses resort to clearing large areas of forest allocate land for cultivation. This also destroys animals' natural feeding grounds. Moreover, such unsustainable practices destroy delicate ecosystems\cite{vegenv}, which is a driving factor of the extinction of certain species.

\section{Opposition}
\paragraph{} Even though these glaring issues are relevant and still will be for some more time to come, it it to be noted that the idea of vegetarianism, in its most frugal manner is an idea that does good not only to the people that adopt it, but also to the ecosystem in general. Flexitarianism is a very good example, in the sense that it allows for the consumption of meat in lower quantities; about once a week or so. This plugs the most of the deficiencies brought about by a ill-advised vegetarian diet. It should still be noted that this is an ideal that is not the case with most of the population that adopts some form of vegetarianism. Also, while the consumption of plant-based foods reduces the requirement of livestock, and thereby contributes to lesser greenhouse gases, it still takes the inherent balance in the food chain and the ecosystem altogether, and this cannot be avoided without a serious revamp to the current agricultural systems.

\section{Conclusion}
\paragraph{} To sum it all up, we should understand that while vegetarianism in itself is a good alternative to most dietary regimes that we pursue today, the concept of vegetarianism is given much more credit than it deserves, simply because in its current incarnation, it poses many issues that are looked over. The health problems of a ill-advised vegan diet, the questions raised against the opposing faction which are blown out of proportion and the environmental concerns are something everyone needs to be aware of before they assess their choices.

\printbibliography

\end{document}

